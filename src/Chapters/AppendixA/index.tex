\chapter{Code}\label{ch:code}
Here we report the code, written in Qiskit and executed on the device.

But what is Qiskit?
Qiskit is an open-source software development kit (SDK) for working with quantum computers at the level of circuits, pulses, and algorithms. It provides tools for creating and manipulating quantum programs and running them on prototype quantum devices on IBM Quantum Experience or on simulators on a local computer. 

It follows the circuit model for universal quantum computation, and can be used for any quantum hardware (currently supports superconducting qubits and trapped ions) that follows this model. 

Note that even if this SDK is developed and maintained by IBM it is fully Open Source and can be used to work in systems that are not IBM's (if they implement support to the language).


\begin{minted}[breaklines, linenos]{python}
    import numpy as np
    import matplotlib.pyplot as plt
    plt.rcParams.update({'font.size': 16})  # enlarge matplotlib fonts
    
    # Import qubit states Zero (|0>) and One (|1>), and Pauli operators (X, Y, Z)
    from qiskit.opflow import Zero, One, I, X, Y, Z
    
    # Suppress warnings
    import warnings
    warnings.filterwarnings('ignore')

    def H_heis3():
    # Interactions (I is the identity matrix; X, Y, and Z are Pauli matricies; ^ is a tensor product)
    XXs = (I^X^X) + (X^X^I)
    YYs = (I^Y^Y) + (Y^Y^I)
    ZZs = (I^Z^Z) + (Z^Z^I)
    
    # Sum interactions
    H = XXs + YYs + ZZs
    
    # Return Hamiltonian
    return H

    # Returns the matrix representation of U_heis3(t) for a given time t assuming an XXX Heisenberg Hamiltonian for 3 spins-1/2 particles in a line
def U_heis3(t):
    # Compute XXX Hamiltonian for 3 spins in a line
    H = H_heis3()
    
    # Return the exponential of -i multipled by time t multipled by the 3 spin XXX Heisenberg Hamilonian 
    return (t * H).exp_i()

    # Define array of time points
ts = np.linspace(0, np.pi, 100)

# Define initial state |110>
initial_state = One^One^Zero

# Compute probability of remaining in |110> state over the array of time points
 # ~initial_state gives the bra of the initial state (<110|)
 # @ is short hand for matrix multiplication
 # U_heis3(t) is the unitary time evolution at time t
 # t needs to be wrapped with float(t) to avoid a bug
 # (...).eval() returns the inner product <110|U_heis3(t)|110>
 #  np.abs(...)**2 is the modulus squared of the innner product which is the expectation value, or probability, of remaining in |110>
probs_110 = [np.abs((~initial_state @ U_heis3(float(t)) @ initial_state).eval())**2 for t in ts]

# Plot evolution of |110>
plt.plot(ts, probs_110)
plt.xlabel('time')
plt.ylabel(r'probability of state $|110\rangle$')
plt.title(r'Evolution of state $|110\rangle$ under $H_{Heis3}$')
plt.grid()
plt.show()

# Importing standard Qiskit modules
from qiskit import QuantumCircuit, QuantumRegister, IBMQ, execute, transpile
from qiskit.providers.aer import QasmSimulator
from qiskit.tools.monitor import job_monitor
from qiskit.circuit import Parameter

# Import state tomography modules
from qiskit.ignis.verification.tomography import state_tomography_circuits, StateTomographyFitter
from qiskit.quantum_info import state_fidelity

# suppress warnings
import warnings
warnings.filterwarnings('ignore')

# load IBMQ Account data

IBMQ.save_account(TOKEN)  # replace TOKEN with your API token string (https://quantum-computing.ibm.com/lab/docs/iql/manage/account/ibmq)
provider = IBMQ.load_account()

from qiskit.providers.aer import AerSimulator# Get backend for experiment
provider = IBMQ.get_provider(hub='ibm-q-community', group='ibmquantumawards', project='open-science-22')
jakarta = provider.get_backend('ibmq_jakarta')
properties = jakarta.properties()
# Simulated backend based on ibmq_jakarta's device noise profile
sim_noisy_jakarta = QasmSimulator.from_backend(provider.get_backend('ibmq_jakarta'))

# Noiseless simulated backend
sim = QasmSimulator()

t = Parameter('t')

Solution_qr = QuantumRegister(2)
Solution_qc = QuantumCircuit(Solution_qr, name='3-CNOT')

Solution_qc.cnot(0,1)
Solution_qc.rx(2*t - np.pi/2,0)
Solution_qc.rz(2*t,1)
Solution_qc.h(0)
Solution_qc.cnot(0,1)
Solution_qc.h(0)
Solution_qc.rz(2*t, 1).inverse()
Solution_qc.cnot(0,1)
Solution_qc.rx(np.pi/2, 0)
Solution_qc.rx(np.pi/2, 1).inverse()

Solution = Solution_qc.to_instruction()

# Combine subcircuits into a single multiqubit gate representing a single trotter step
num_qubits = 3

Trot_qr = QuantumRegister(num_qubits)
Trot_qc = QuantumCircuit(Trot_qr, name='Trot')

for i in range(0, num_qubits - 1):
    Trot_qc.append(Solution, [Trot_qr[i], Trot_qr[i+1]])

# Convert custom quantum circuit into a gate
Trot_gate = Trot_qc.to_instruction()

# The final time of the state evolution
target_time = np.pi

# Number of trotter steps
trotter_steps = 4  ### CAN BE >= 4

# Initialize quantum circuit for 3 qubits
qr = QuantumRegister(7)
qc = QuantumCircuit(qr)

# Prepare initial state (remember we are only evolving 3 of the 7 qubits on jakarta qubits (q_5, q_3, q_1) corresponding to the state |110>)
qc.x([3,5])  # DO NOT MODIFY (|q_5,q_3,q_1> = |110>)

# Simulate time evolution under H_heis3 Hamiltonian
for _ in range(trotter_steps):
    qc.append(Trot_gate, [qr[1], qr[3], qr[5]])

# Evaluate simulation at target_time (t=pi) meaning each trotter step evolves pi/trotter_steps in time
qc = qc.bind_parameters({t: target_time/trotter_steps})

# Generate state tomography circuits to evaluate fidelity of simulation
st_qcs = state_tomography_circuits(qc, [qr[1], qr[3], qr[5]])

# Display circuit for confirmation
#st_qcs[-1].decompose().decompose().draw('latex')  # view decomposition of trotter gates
st_qcs[-1].draw('mpl')  # only view trotter gates

shots = 8192
reps = 8
backend = sim_noisy_jakarta
config = backend.configuration()
# reps = 8
# backend = jakarta

jobs = []
for _ in range(reps):
    # execute
    job = execute(st_qcs, backend, shots=shots)
    print('Job ID', job.job_id())
    jobs.append(job)

print("This backend is called {0}, and is on version {1}. It has {2} qubit{3}. It "
      "{4} OpenPulse programs. The basis gates supported on this device are {5}."
      "".format(config.backend_name,
                config.backend_version,
                config.n_qubits,
                '' if config.n_qubits == 1 else 's',
                'supports' if config.open_pulse else 'does not support',
                config.basis_gates))

                for job in jobs:
    job_monitor(job)
    try:
        if job.error_message() is not None:
            print(job.error_message())
    except:
        pass

        # Compute the state tomography based on the st_qcs quantum circuits and the results from those ciricuits
def state_tomo(result, st_qcs):
    # The expected final state; necessary to determine state tomography fidelity
    target_state = (One^One^Zero).to_matrix()  # DO NOT MODIFY (|q_5,q_3,q_1> = |110>)
    # Fit state tomography results
    tomo_fitter = StateTomographyFitter(result, st_qcs)
    rho_fit = tomo_fitter.fit(method='lstsq')
    # Compute fidelity
    fid = state_fidelity(rho_fit, target_state)
    return fid

# Compute tomography fidelities for each repetition
fids = []
for job in jobs:
    fid = state_tomo(job.result(), st_qcs)
    fids.append(fid)
    
print('state tomography fidelity = {:.4f} \u00B1 {:.4f}'.format(np.mean(fids), np.std(fids)))
\end{minted}