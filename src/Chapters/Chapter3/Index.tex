\chapter{The challenge}\label{chap:3}
\epigraph{ Provando e riprovando.}{Accademia del Cimento, Florence, 1675.}
Here we start explaining the actual competition. First of all, what we have to do it is simply simulate an Hamiltonian as explained in chapter~\ref{chap:2}, nothing more. The Hamiltonian will be presented shortly in \ref{sec:xxx}. There are however some restriction:
\begin{enumerate}
    \item Trotterization must be used in the decomposition and the number of steps should be greater or equal to four.
    \item The simulation will be tested and executed on a specific device, Jakarta, that we will describe in \ref{sec:jakarta}
\end{enumerate}

Note that the access to the real device is tied to a long (very long!) queue and time of execution itself is quite long too. That's why IBM provide backend simulators, i.e. classical computer that simulates specifically quantum backends.
Of course we will use \mintinline[breaklines]{python}{sim_noisy_jakarta = QasmSimulator.from_backend(provider.get_backend('ibmq_jakarta'))} which is the simulator of Jakarta on a classical device. Of course simulations are not as near as precise as executions on the real device, however they can guide you towards the right direction, saving execution on the real device for more fine-grained refinements.

There are even backends simulators without noise, however they were not used as, after some testing, they seemed completely out of touch with the results of the real device.
\section{The XXX Model}\label{sec:xxx}

The quantum Heisenberg model, developed by Werner Heisenberg, is a statistical mechanical model used in the study of critical points and phase transitions of magnetic systems, in which the spins of the magnetic systems are treated quantum mechanically. The model is based on the Hamiltonian:
    \begin{equation}
        H = -1/2 \sum_{j=1}^{N} (J_x \sigma_x^{j} \sigma_x^{j+1} + J_y \sigma_y^{j} \sigma_y^{j+1} + J_z \sigma_z^{j} \sigma_z^{j+1})
     \end{equation}

     where $J$ is the coupling constant. We will use a simplified version. This version of the general Heisenberg spin model is called $XXX$ because the same $J$ value multiplies each pair of Pauli operators.




     \begin{equation}
        H_{\text{Heis}} = \sum_{\langle ij \rangle}^{N} J \left(\sigma_x^{(i)}\sigma_x^{(j)} + \sigma_y^{(i)}\sigma_y^{(j)} + \sigma_z^{(i)}\sigma_z^{(j)}\right).
     \end{equation}
    
    
    $N$ is the number of spin-1/2 particles in model.The $i$ and $j$ superscripts label which qubit they act on. For example, $\sigma_x^{(1)}$ would be the $\sigma_x$ operator acting on only qubit 1 (which is the 2nd qubit since indexing starts at 0). The sum notation $\langle ij \rangle$ means the sum is over nearest neighbors (only qubits next to each other interact), and $J$ is the interaction strength, which we will set $J=1$.

    

    Of course $\sigma_x, \sigma_y, \sigma_z$ are the known pauli matrices: 

\[
    \sigma_x =
    \begin{pmatrix}0&1\\1&0\end{pmatrix}
    \]
    \[
    \sigma_y =
    \begin{pmatrix}0&-i\\i&0\end{pmatrix} 
    \]
    \[
    \sigma_z =
    \begin{pmatrix}1&0\\0&-1\end{pmatrix} 
    \]


    We will work with the explicit case of $N=3$ with the 3 spins arranged in a line. Written out fully, the Hamiltonian is
\begin{equation}
\begin{split}
H_{\text{Heis3}} &= \sigma_x^{(0)}\sigma_x^{(1)} + \sigma_x^{(1)}\sigma_x^{(2)} + \sigma_y^{(0)}\sigma_y^{(1)} \\
&+ \sigma_y^{(1)}\sigma_y^{(2)} + \sigma_z^{(0)}\sigma_z^{(1)} + \sigma_z^{(1)}\sigma_z^{(2)}.
\end{split}
\end{equation}
Now that we have a Hamiltonian ($H_{\text{Heis3}}$), we can use it to determine how the quantum system of 3 spin-1/2 particles changes in time.


To compute the matrix representation of $H_{\text{Heis3}}$, we are actually missing some pieces namely the identity operator $I$ and the tensor product $\otimes$ symbol. They are both often left out in when writing a Hamiltonian, but they are implied to be there. 

Writing out the full $H_{\text{Heis3}}$ including the identity operators and tensor product symbols we obtain: 

\begin{equation}
    \begin{split}
H_{\text{Heis3}} &= \sigma_x^{(0)}\otimes\sigma_x^{(1)}\otimes I^{(2)} + I^{(0)} \otimes\sigma_x^{(1)}\otimes\sigma_x^{(2)} \\ &+ \sigma_y^{(0)}\otimes\sigma_y^{(1)}\otimes I^{(2)} + 
I^{(0)} \otimes \sigma_y^{(1)}\otimes\sigma_y^{(2)} \\ &+ I^{(0)} \otimes\sigma_z^{(0)}\otimes\sigma_z^{(1)} + I^{(0)}\otimes\sigma_z^{(1)}\otimes\sigma_z^{(2)}.
\end{split}
\end{equation}


Knowing the Hamiltonian, we can determine how quantum states of that system evolve in time by solving the Schrödinger equations
\begin{equation}
i\hbar \dfrac{d}{dt}|\psi(t)\rangle = H |\psi(t)\rangle
\end{equation}

For simplicity, let's set $\hbar = 1$. We know that the Hamiltonian $H_{\text{heis3}}$ does not change in time, so the solution to the Schrödinger equation is an exponential of the Hamiltonian operator

\begin{align}
U_{\text{Heis3}}(t) &= e^{-it H_\text{Heis3}} = \exp\left(-it H_\text{Heis3}\right) \\
U_{\text{Heis3}}(t) &= \exp\left[-it \sum_{\langle ij \rangle}^{N=3} \left(\sigma_x^{(i)}\sigma_x^{(j)} + \sigma_y^{(i)}\sigma_y^{(j)} + \sigma_z^{(i)}\sigma_z^{(j)}\right) \right]
\end{align}



\begin{equation}
    \begin{split}
        U_{\text{Heis3}}(t) &= \exp\left[-it \left(\sigma_x^{(0)}\sigma_x^{(1)} + \sigma_x^{(1)}\sigma_x^{(2)} + \sigma_y^{(0)}\sigma_y^{(1)}  + \sigma_y^{(1)}\sigma_y^{(2)} + \sigma_z^{(0)}\sigma_z^{(1)} + \sigma_z^{(1)}\sigma_z^{(2)}\right) \right]
    \end{split}
\end{equation}
Now that we have the time evolution operator $U_{\text{Heis3}}(t)$, we can simulate changes in a state of the system ($|\psi(t)\rangle$) over time $|\psi(t)\rangle = U_{\text{Heis3}}(t)|\psi(t=0)\rangle$. 

Our goal will be to decompose $U_{\text{Heis3}}(t)$ into one and two-qubit gates executable in the Jakarta device.
\section{Qiskit Pulse and native gates}
Something that we have still not explained is how to implement our decomposition in the machine.
First of all, our code is written in Qiskit, a Python SDK, that we will describe better in appendix~\ref{ch:code}.
However tha challenge give us two possibilities:
\begin{enumerate}
\item Using native gates.
\item Using Qiskit Pulse.
\end{enumerate}

In the first case, we simply call the native gates with the predefined commands\footnote{Of course we can change parameters, phase and so on.} and so we are forced to limit our decomposition to the available gates.

Qiskit Pulse on the other ands offers low-level control of a device's qubits. Pulse allows users to program the physical interactions happening on the superconducting chip. This can be a powerful tool for streamlining circuits, crafting new types of gates, getting higher fidelity readout, and more.

We are going to use the first strategy, as it is the most straightforward to implement for a single person working on the project.

\section{State of the art}
\subsection{Different qubit implementations}
\section{The solution}
We want to decompose 
\begin{equation}
    U_{\text{Heis3}}(t) = \exp\left[-it \sum_{\langle ij \rangle}^{N=3} \left(\sigma_x^{(i)}\sigma_x^{(j)} + \sigma_y^{(i)}\sigma_y^{(j)} + \sigma_z^{(i)}\sigma_z^{(j)}\right) \right]
    \end{equation}
into single and two-qubit gates.

Since the Pauli operators do not commute with each other~\cite{Shankar} the exponential $U_{\text{Heis3}}(t)$ cannot be split into a product of simpler exponentials.

However, we can approximate $U_{\text{Heis3}}(t)$ as a product of simpler exponentials through Trotterization. Consider a subsystem of 2 spin-1/2 particles within the larger 3 spin system. The Hamiltonian on spins $i$ and $j$ ($i,j \in \{0,1,2\}$) would be $H^{(i,j)}_{\text{Heis2}} = \sigma_x^{(i)}\sigma_x^{(j)} + \sigma_y^{(i)}\sigma_y^{(j)} + \sigma_z^{(i)}\sigma_z^{(j)}$. Rewriting $U_{\text{Heis3}}(t)$ in terms of the two possible subsystems within the total $N=3$ system you will simulate,

\begin{equation}
U_{\text{Heis3}}(t) = \exp\left[-i t \left(H^{(0,1)}_{\text{Heis2}} + H^{(1,2)}_{\text{Heis2}} \right)\right].
\end{equation}

$H^{(0,1)}_{\text{Heis2}}$ and $H^{(1,2)}_{\text{Heis2}}$ do not commute, so
\begin{equation}
 U_{\text{Heis3}}(t) \neq \exp\left(-i t H^{(0,1)}_{\text{Heis2}}\right) \exp\left(-i t H^{(1,2)}_{\text{Heis2}} \right)
\end{equation}. 
 
 But, this product decomposition can be approximated with Trotterization which says $U_{\text{Heis3}}(t)$ is approximately a short evolution of $H^{(0,1)}_{\text{Heis2}}$ (time = $t/n$) and followed by a short evolution of $H^{(1,2)}_{\text{Heis2}}$ (time = $t/n$) repeated $n$ times


\begin{align}
U_{\text{Heis3}}(t) &= \exp\left[-i t \left(H^{(0,1)}_{\text{Heis2}} + H^{(1,2)}_{\text{Heis2}} \right)\right] \\
U_{\text{Heis3}}(t) &\approx \left[\exp\left(\dfrac{-it}{n}H^{(0,1)}_{\text{Heis2}}\right) \exp\left(\dfrac{-it}{n}H^{(1,2)}_{\text{Heis2}} \right)\right]^n.
\end{align}


$n$ is the number of Trotter steps, and as $n$ increases, the approximation should becomes more accurate but as we have already anticipated experimentally this not seems to be the case.

But now we have a state of the art solution for decomposing \text{Heis2} into quantum gates~\cite{s1} \cite{s2} and this is shown in figure~\ref{fig:solution}, we can then implement this decomposition of \text{Heis2} into our decomposition of \text{Heis3}.

In the decomposition of Heis2 we have:

\begin{equation}
    R X(\theta)=\exp \left(-i \frac{\theta}{2} X\right)=\left(\begin{array}{cc}
        \cos \frac{\theta}{2} & -i \sin \frac{\theta}{2} \\
        -i \sin \frac{\theta}{2} & \cos \frac{\theta}{2}
        \end{array}\right)
\end{equation}

and 
\begin{equation}
    R Z(\lambda)=\exp \left(-i \frac{\lambda}{2} Z\right)=\left(\begin{array}{cc}
        e^{-i \frac{\lambda}{2}} & 0 \\
        0 & e^{i \frac{\lambda}{2}}
        \end{array}\right).
\end{equation}

Those are not native gates but as already said they can be implemented as Jakarta has a universal set of quantum gates, and we can change the phase factor directly without the need of using Pulse.
\begin{figure}[htb]
    \includegraphics[width = 1.2\textwidth]{circuit.png}
    \centering
    \caption{Heis2 decomposition}
    \label{fig:solution}
\end{figure}


At the end our circuit is represented in figure~\ref{fig:circuit} where we can see the 7 qubits, the 4 trotterization steps and the measure at the end. The circuit of~\ref{fig:solution} is contained in each trotterization step. 
\begin{figure}[htb]
    \includegraphics[width = 1.2\textwidth]{output1.png}
    \centering
    \caption{Circuit overview}
    \label{fig:circuit}
\end{figure}

Finally, those are the results:
\begin{description}
    \item[Noisy simulation: ]0.4405 ± 0.0011 (N=4)
    \item[Real device: ]0.3108 ± 0.0034 (N=4)
    \end{description}

    You may ask why there were not tried more complicated decomposition, the reason is that at the end reducing the original Hamiltonian to Hamiltonians with already recognized state of the art decomposition resulted to be the strategy that provided the better results, as completely raw decomposition directly to gates resulted in significantly weaker results.

    In addition to that the least number of trotterization steps as the problem required were used. This is because, contrary to what Lie's formula presented in chapter~\ref{chap:2} seems to suggests, experimentally it was very clear that augmenting trotterization steps significantly reduced state tomography results, probably because no noise suppression techniques were used in order to deal with the the fact that the circuit increases with more Trotterization steps.

