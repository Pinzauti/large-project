\section{State tomography}
How we check if the result of our simulation is correct? State tomography is a method for determining the quantum state of a qubit, or qubits, even if the state is in a superposition or entangled. Repeatedly measuring a prepared quantum state may not be enough to determine the full state, in state tomography, a quantum circuit is repeated with measurements done in different bases to exhaustively determine the full quantum state (including any phase information). IBM is using this technique to determine the full quantum state after the quantum simulation. That state is then compared to the exact expected result to compute a fidelity. Although this fidelity only gives information on how well the quantum simulation produces one particular state, it's a more lightweight approach than a full process tomography calculation. In short, a high fidelity measured by state tomography doesn't guarente a high fidelity quantum simulation, but a low fidelity state tomography does imply a low fidelity quantum simulation. 

In our problem the tomography is done on qubit 1,3,5 (more on that in chapter~\ref{chap:3}). The result is a number between 0 and 1 with 1 meaning a perfect result. Error mitigation can be
applied before the state tomography according to the rules.~\cite{Tomography}
